\documentclass{article}
\textwidth=14cm
\usepackage[utf8]{inputenc}
\usepackage[spanish]{babel}
\usepackage{fancyhdr}
\usepackage{amssymb}
%\usepackage{pythonhighlight}
\usepackage[pdftex]{hyperref}
\usepackage{graphicx} % Inclusión de imágenes


\hypersetup{colorlinks=true,linkcolor=black}

%%%%%scabecera
\pagestyle{fancy} % seleccionamos un estilo
\lhead{UNPSJB } % texto izquierda de la cabecera
%\chead{TEXTO} % texto centro de la cabecera
\chead{Sistemas de Soporte para la Toma de Decisiones}% \includegraphics[width=2cm]{./unpsjb.png}} % número de página a la derecha
%\lfoot{TEXTO} % texto izquierda del pie
\rhead{\includegraphics[scale = 0.15]{Imagenes/unpsjb.png}} % imagen centro del pie
%\rfoot{TEXTO} % texto derecha del pie
\renewcommand{\headrulewidth}{0.4pt} % grosor de la línea de la cabecera
\renewcommand{\footrulewidth}{0.4pt} % grosor de la línea del pie

\begin{document}
\begin{large}
\title{\bf {Trabajo Práctico Nro 4 - Sistemas de Soporte para la Toma de Decisiones.}}
\end{large}
\date{2013 - UNPSJB\\[1cm]\includegraphics[scale = 0.5]{Imagenes/unpsjb.png}}

\maketitle
\newpage
%\begin{figure}[h]
%\begin{center}
%\includegraphics[scale=0.50]{./Imagenes/graf1.png}
%\label{fig:graf}
%\caption{}
%\end{center}
%\end{figure}
%Poner Imagen aqui

%Data warehousing logical design
%Mirjana Mazuran
%mazuran@elet.polimi.it
%December 15, 2009
Alguna intro ...

\section{Mineria de Datos sobre Redes Sociales y Sitios Web}
Los ejercicios deben ser entregados en formato \emph{ipynb} en la fecha pactada por la cátedra.

%\subsection{Mineria de Datos sobre Redes Sociales y Sitios Web}

\begin{enumerate}
\item Tomando datos de la red social Twitter, al menos 200 tweets, realice un script que recolecte, procese y vizualice los datos con las siguientes restricciones:
\begin{enumerate}
	\item Los tres \emph{trends} más \emph{RT} del momento.
	\item Listar nombres de usuarios que publicaron con los \emph{hashtag} de los \emph{trends} del punto anterior.
	\item Cuales son las cinco palabras más utilizadas en los \emph{tweets} del primer item.
	\item Listar los primeros 10 usuarios con mayor cantidad de seguidores. 	
	\item Listar la ubicación (o en su defecto \emph{time-zone}) del \emph{tweet}.
	\item Listar los cinco \emph{tweets} más populares.
	\item Listar a los seguidores del autor del \emph{tweet} más popular.
\end{enumerate}
%\begin{figure}[h]
%\begin{center}
%\includegraphics[scale=0.7]{./Imagenes/twt.png}
%\label{fig:graf2}
%\caption{Cómo esta distribuida la información en un Objeto del Status de Twitter?}
%\end{center}%
%\end{figure}
\item Tomando datos de la red Social Google+, recolecte, procese y vizualice los datos con las siguientes restricciones:
\begin{enumerate}
	\item Obtener las tres últimas actividades de X usuario. 
	\item Analizar el texto de una de estas actividades. (Palabras utilizadas, cantidad de veces, etc)
	\item Visualizar el avatar de tres individuos a seleccionar.
\end{enumerate}
\item Sacando información de Microformatos en la Web.
\begin{enumerate}
	\item Obtener y visualizar coordenadas geográficas.
	\item Listar calendario de eventos.
\end{enumerate}
\end{enumerate}
\begin{enumerate}
\item Programe un Script que determine mediante una Máquina de vectores de soporte (SVMs) si el postulante detallado en \emph{Gonzalez\_prestamo.txt} es apto para aprobación de un crédito y si es así, cual sería el monto habilitado.

\end{enumerate}
\end{document}
