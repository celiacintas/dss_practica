\documentclass{article}
\textwidth=14cm
\usepackage[utf8]{inputenc}
\usepackage[spanish]{babel}
\usepackage{fancyhdr}
\usepackage{amssymb}
%\usepackage{pythonhighlight}
\usepackage[pdftex]{hyperref}
\usepackage{graphicx} % Inclusión de imágenes


\hypersetup{colorlinks=true,linkcolor=black}

%%%%%scabecera
\pagestyle{fancy} % seleccionamos un estilo
\lhead{UNPSJB } % texto izquierda de la cabecera
%\chead{TEXTO} % texto centro de la cabecera
\chead{Sistemas de Soporte para la Toma de Decisiones}% \includegraphics[width=2cm]{./unpsjb.png}} % número de página a la derecha
%\lfoot{TEXTO} % texto izquierda del pie
\rhead{\includegraphics[scale = 0.15]{Imagenes/unpsjb.png}} % imagen centro del pie
%\rfoot{TEXTO} % texto derecha del pie
\renewcommand{\headrulewidth}{0.4pt} % grosor de la línea de la cabecera
\renewcommand{\footrulewidth}{0.4pt} % grosor de la línea del pie

\begin{document}
\begin{large}
\title{\bf {Trabajo Práctico Nro 2 - Sistemas de Soporte para la Toma de Decisiones.}}
\end{large}
\date{2013 - UNPSJB\\[1cm]\includegraphics[scale = 0.5]{Imagenes/unpsjb.png}}

\maketitle
\newpage
%\begin{figure}[h]
%\begin{center}
%\includegraphics[scale=0.50]{./Imagenes/graf1.png}
%\label{fig:graf}
%\caption{}
%\end{center}
%\end{figure}
%Poner Imagen aqui

%Data warehousing logical design
%Mirjana Mazuran
%mazuran@elet.polimi.it
%December 15, 2009

Un DSS puede adoptar muchas formas diferentes. En general, podemos decir que un DSS es un sistema informático utilizado para servir de apoyo, más que automatizar, el proceso de toma de decisiones. La decisión es una elección entre alternativas basadas en estimaciones de los valores de esas alternativas. El apoyo a una decisión significa ayudar a las personas que trabajan solas o en grupo a reunir inteligencia, generar alternativas y tomar decisiones. Apoyar el proceso de toma de decisión implica el apoyo a la estimación, la evaluación y/o la comparación de alternativas. En la práctica, las referencias a DSS suelen ser referencias a aplicaciones informáticas que realizan una función de apoyo.

\section{Tipos de DSS}
A continuación enumeramos algunos ejemplos de Sistemas de apoyo para la toma de decisiones:
\begin{description}
    \item[California Pizza Kitchen] La aplicación \emph{Express Inventory} recuerda los patrones de pedido
de cada restaurante y compara la cantidad de ingredientes usados por elementos de menú con las medidas de porciones predefinidas establecidas por la administración. El sistema identifica los restaurantes con porciones que no se apegan a las normas y notifica a su administración para que pueda tomarse la acción correctiva. %MIS
    \item[Taco Bell] El sistema TACOA (automatización Total de la operaciones de la empresa) proporciona información sobre  alimentos, mano de obra y costo periódico para cada restaurante. %MIS
    \item[Continental Airlines Inc.] Sistema para la optimización de ingresos por carga de Continental Airlines Inc. La división de carga de Continental desarrollo una aplicación llamada \emph{cargoprof} para maximizar los ingresos provenientes de los comportamientos de carga de sus navíos. Se trata de un paquete personalizado de \emph{Manugistrics Inc. } de Richville, Maryland, y asegura que \emph{Continental} venda todo el espacio disponible de carga en sus aviones el precio mas rentable. El sistema pronostica la capac idad de carga y establecer un valor óptimo cada noche sobre lo que necesitan. %DSS
    \item[Homes.Com] Proporciona un listado de casa en venta, apartamentos en renta e hipotecas disponibles en estados
unidos. Los visitantes pueden averiguar para qué hipotecas califican y calcular la hipoteca máxima que se pueden permitir y los
pagos alternativos mensuales de la hipoteca. También pueden usar las herramientas para determinar si deben alquilar o comparar. %CDSS
\end{description}

\begin{enumerate}
    \item Clasifique los distintos Sistemas según la clasificación vista en la teoría.\footnote{Dirigidos por Modelos, por Conocimiento, por Datos, Sistemas de Soporte para la Toma de Decisiones de Grupo, Sistemas de Información para Ejecutivos.}
    \item Qué datos se necesitan para los ejemplos enumerados anteriormente? Se necesitan datos externos?
    \item En cada caso que datos y de que forma se le presentaría.
	\item Indicar si hace falta inteligencia adicional del sistema además de la reunión y clasificación de información.
\end{enumerate}
\end{document}
