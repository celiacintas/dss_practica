\documentclass{article}
\textwidth=14cm
\usepackage[utf8]{inputenc}
\usepackage[spanish]{babel}
\usepackage{fancyhdr}
\usepackage{amssymb}
%\usepackage{pythonhighlight}
\usepackage[pdftex]{hyperref}
\usepackage{graphicx} % Inclusión de imágenes


\hypersetup{colorlinks=true,linkcolor=black}

%%%%%scabecera
\pagestyle{fancy} % seleccionamos un estilo
\lhead{UNPSJB } % texto izquierda de la cabecera
%\chead{TEXTO} % texto centro de la cabecera
\chead{Sistemas de Soporte para la Toma de Decisiones}% \includegraphics[width=2cm]{./unpsjb.png}} % número de página a la derecha
%\lfoot{TEXTO} % texto izquierda del pie
\rhead{\includegraphics[scale = 0.15]{Imagenes/unpsjb.png}} % imagen centro del pie
%\rfoot{TEXTO} % texto derecha del pie
\renewcommand{\headrulewidth}{0.4pt} % grosor de la línea de la cabecera
\renewcommand{\footrulewidth}{0.4pt} % grosor de la línea del pie

\begin{document}
\begin{large}
\title{\bf {Trabajo Práctico Nro 1 - Las Organizaciones y las Decisiones.}}
\end{large}
\date{2013 - UNPSJB\\[1cm]\includegraphics[scale = 0.5]{Imagenes/unpsjb.png}}

\maketitle
\newpage
%\begin{figure}[h]
%\begin{center}
%\includegraphics[scale=0.50]{./Imagenes/graf1.png}
%\label{fig:graf}
%\caption{}
%\end{center}
%\end{figure}
%Poner Imagen aqui

%Data warehousing logical design
%Mirjana Mazuran
%mazuran@elet.polimi.it
%December 15, 2009

La toma de decisiones en las empresas solía ser responsabilidad exclusiva de la administración. En la actualidad empleados de niveles inferiores son responsables de algunas decisiones, a medida que los sistemas de información ponen esta a disposición de los niveles inferiores de la empresa ¿Cómo se realiza la toma de decisiones de las empresas y en organizaciones?¿Cómo identificamos las diferentes decisiones y sus tipos?

\section{Tipos de Decisiones}
A continuación enumeramos algunas decisiones típicas:
\begin{itemize}
    \item Aprobación de presupuestos de capital.
    \item Reabastecer el inventario.
    \item Decidir $X$ meta a largo plazo.
    \item Diseño de plan de \emph{Marketing}.
    \item Desarrollo de presuspuesto de $Y$ departamento.
    \item Ofrecer créditos a clientes.
    \item Diseño de sitio Web.
    \item Determinar ofertas especiales para cierto segmento del mercado.
\end{itemize}
\begin{enumerate}
    \item Clasifique las distintas decisiones en sus tipos correspondientes\footnote{No estructuradas, estructuradas, semiestructuradas}.
    \item El tipo de decisión esta siempre relacionado con los niveles de una organización? Justifique.
\end{enumerate}

\section{Roles Administrativos}
A continuación enumeramos acciones clásicas del rol administrativo:
\begin{itemize}
    \item Determinación de calendarios. %rol decision
    \item Planeamiento de presupuestos. % rol decision
    \item Reuniónes periódicas con Clientes y Proveedores. % rol interpersonales
    \item Creación de Juntas de consejo. % rol informativo
    \item Revisión de estrategias. % rol decision
    \item Sesiones de revisión y generación de boletines. %rol informativo
    \item Negociación en conflictos. % rol decision
    \item Administración  y actualización de contactos en orden de relevancia. %rol informativo
    \item Organización de eventos sociales, motivación de empleados. % rol interpersonales
\end{itemize}
\begin{enumerate}
    \item Identifique e indique que tipo de rol\footnote{Roles Interpersonales, Roles Informativos, Roles de Decisión} estaría realizando cada una de estas acciones.
    \item Enumere ejemplos de sistemas de apoyo para cada rol.
\end{enumerate}

\section{Caso de Estudio}

\begin{enumerate}
    \item Del siguiente caso de estudio evalue:
    \begin{enumerate}
        \item Rol que toma el director general.
        \item Cuales son las decisiones que aparecen en este caso y cual es su clasificación?
    \end{enumerate}
    \item Enumere ventajas operativas y estratégicas que puede obtener la empresa \emph{Transportes Veracruzanos}
con base en el desarrollo de sus sistemas de decisiones o procesos decisorios.
    \item Describa que sistemas de decisiones desarrollaría para ayudar a la toma de decisiones enumeradas anteriormente.

\end{enumerate}
La empresa \emph{Transportes Veracruzanos}, se dedica al transporte de
carga pesada. La empresa cuenta con oficinas autónomas en las ciudades de México, Poza
Rica, Minatitlán y Puebla. El operativo reside en Veracruz, donde está la dirección general
y las aéreas de apoyo como la gerencia de operaciones, mantenimiento, investigación y
desarrollo, la contraloría general y la gerencia de recursos humanos. Estas divisiones se
encargan de definir las estrategias de negocio tanto para el corporativo como para las
oficinas distantes. Entre los sistemas que se han implantado en cada una de las oficinas se
pueden mencionar el control de tráfico, el sistema de mantenimiento a unidades, control de
llantas, el sistema de facturación y cobranza, cuentas por pagar, la nomina, el sistema de
compras y, por supuesto, el sistema contable.\\

 Para el director general es de suma
importancia conocer la situación operativa y financiera de la empresa en tiempo real, es
decir tener información al día de cada una de las unidades de negocio, pues es el quien
rinde información al consejo administrativo. De aquí que la consolidación de las unidades
de negocio (oficinas remotas) debe realizarse por lo menos mensualmente.
El director general reporta cada trimestre al consejo administrativo, en esas reuniones se
analiza y definen estrategias y políticas a seguir.\\

Para lograr la consolidaciones necesario contar con una infraestructura de comunicación,
que en este caso consiste en la contratación de líneas privadas con la compañía nacional de
teléfono, enlaces mediante los cuales Transportes Veracruzanos logro establecer un flujo de
información entre oficinas remotas y el corporativo.\\

Uno de los problemas que enfrenta la empresa es que sus aplicaciones administrativas
(sistemas) no operan con el apoyo de un DBMS. Por tanto, la empresa implanto un
programa para redefinir su plataforma de sistemas en una arquitectura cliente-servidor.
En cuanto al desarrollo de las aplicaciones existe la disyuntiva de desarrollar los programas
de escritorio o aplicaciones Web. Existen algunas limitaciones económicas para
desarrollar el proyecto, pues los presupuestos de ventas de los últimos seis meses no se han
cumplido.

\end{document}
