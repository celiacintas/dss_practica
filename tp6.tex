\documentclass{article}
\textwidth=14cm
\usepackage[utf8]{inputenc}
\usepackage[spanish]{babel}
\usepackage{fancyhdr}
\usepackage{amssymb}
%\usepackage{pythonhighlight}
\usepackage[pdftex]{hyperref}
\usepackage{graphicx} % Inclusión de imágenes


\hypersetup{colorlinks=true,linkcolor=black}

%%%%%scabecera
\pagestyle{fancy} % seleccionamos un estilo
\lhead{UNPSJB } % texto izquierda de la cabecera
%\chead{TEXTO} % texto centro de la cabecera
\chead{Sistemas de Soporte para la Toma de Decisiones}% \includegraphics[width=2cm]{./unpsjb.png}} % número de página a la derecha
%\lfoot{TEXTO} % texto izquierda del pie
\rhead{\includegraphics[scale = 0.15]{Imagenes/unpsjb.png}} % imagen centro del pie
%\rfoot{TEXTO} % texto derecha del pie
\renewcommand{\headrulewidth}{0.4pt} % grosor de la línea de la cabecera
\renewcommand{\footrulewidth}{0.4pt} % grosor de la línea del pie

\begin{document}
\begin{large}
\title{\bf {Trabajo Práctico Nro 6 - Sistemas de Soporte para la Toma de Decisiones.}}
\end{large}
\date{2013 - UNPSJB\\[1cm]\includegraphics[scale = 0.5]{Imagenes/unpsjb.png}}

\maketitle
\newpage
%\begin{figure}[h]
%\begin{center}
%\includegraphics[scale=0.50]{./Imagenes/graf1.png}
%\label{fig:graf}
%\caption{}
%\end{center}
%\end{figure}
%Poner Imagen aqui

%Data warehousing logical design
%Mirjana Mazuran
%mazuran@elet.polimi.it

Día a día se generan más datos, entenderlos es crucial para poder tomar decisiones acertadas y la
visualización nos permite analizarlos de forma rápida, intuituva y precisa. \textbf{MEJORAR LA INTRO}

\section{Visualización}
\begin{enumerate}
\item Enumere para que tareas es util la visualización computacional.
\item Cuando nuestra herramienta provee Exploración Visual de Datos, qué características son importantes?
%\item Qué diferencias encontramos a la hora de visualizar datos científicos visualización de información?
\item Gráficas de negocio es lo mismo que la visualización de información? Justifique.
\item Cuál es el $pipe$ clásico a la hora de visualizar datos?
\end{enumerate}
\section{Visualización Pasiva}
\begin{enumerate}
\item Adquiera información de los seguidores de $X$ usuario en \emph{Twitter} y realize las siguientes actividades:
\begin{itemize}
	\item Clasifique la adquisición de los datos.
	\item De todos los datos, tome un subconjunto de seguidores del usuario $X$ que cumplan con la condición de al menos 10 seguidores.
	\item Dependiendo de la cantidad de seguidores que tenga $Y$ seguidor, asignele un color.
	\item Eliga el tipo de gráfico que desea utlizar, justifique y visualice.
\end{itemize}
\item Teniendo en cuenta los pasos realizados en el punto anterior (Adquisición, preparación, estructuración pre visualización y visualización). Tome $n$ con $n > 10$  coordenadas geográficas y grafíquelas en un mapa.
\item Descargue datos SRTM y visualize. Detalle los pasos realizados. 
\item Descargue los datos del Registro de Accidentes de Tránsito de Bahia Blanca, y visualice de la mejor forma posible todos sus datos (fecha, tipo de evento, domicilio, localidad, latitud, longitud)
\end{enumerate}

\section{Visualización Activa}

\begin{enumerate}
	\item Agregue al item anterior de Registro de Accidentes de Transito la posibilidad de ocultar o mostrar ciertos datos, muestre valores totales y realize filtros (ver información por fecha, tipo de accidente, zona o combinaciones de esta).
	\item Cargue todos los datos de Compras Municipales de Gobierno Abierto, que filtros y consultas habilitaría para ver en que tópicos se invirtió más dinero y cuales fueron los proovedores más beneficiados.
\end{enumerate}

\section{Steering Vis}
\textbf{Queda para tp final .. ver ipython notebook para links y bibliotecas}


\end{document}

